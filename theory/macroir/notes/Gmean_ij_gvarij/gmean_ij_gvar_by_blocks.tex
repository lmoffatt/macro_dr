\documentclass[11pt]{article}
\usepackage{amsmath,amssymb}

\title{Mean and Variance of Integrated Conductance in a Continuous-Time Markov Chain}
\author{}
\date{}

\begin{document}
\maketitle

\section{Setup}

Let $X_t$ be a continuous-time Markov chain on a finite state space
$\{1,\dots,K\}$ with generator $Q \in \mathbb{R}^{K\times K}$.
We write the state-dependent conductance as a vector
$\boldsymbol{\gamma} = (\gamma_1,\dots,\gamma_K)^\top$
and let $\Gamma = \mathrm{diag}(\boldsymbol{\gamma})$.

Given a time horizon $t>0$ and a sample path of the chain, we define the
\emph{integrated conductance}
\begin{equation}
  A(t)
  \;=\;
  \int_0^t \gamma\bigl(X_s\bigr)\,ds.
\end{equation}
We are interested in the first two moments of $A(t)$, and in particular its
mean and variance, conditional on the chain starting in state $i$ and ending
in state $j$:
\[
  X_0 = i, \qquad X_t = j.
\]

Note that the interval-averaged conductance used in the main text is
\begin{equation}
  \bar\gamma_{i\to j}(t)
  \;=\;
  \frac{1}{t}\,
  \mathbb{E}\!\bigl[A(t)\mid X_0=i,X_t=j\bigr],
\end{equation}
so moments of $\bar\gamma_{i\to j}(t)$ follow directly from moments of $A(t)$
by dividing by powers of $t$.

\section{Tilted generator and Feynman--Kac representation}

Define the \emph{tilted generator}
\begin{equation}
  Q(\alpha) = Q + \alpha\,\Gamma, \qquad \alpha \in \mathbb{R},
\end{equation}
and the corresponding transition matrix
\begin{equation}
  P^{(\alpha)}(t) = e^{Q(\alpha) t}.
\end{equation}
A standard Feynman--Kac argument for continuous-time Markov chains yields
\begin{equation}
  P^{(\alpha)}_{ij}(t)
  =
  \mathbb{E}\Big[
    \exp\!\Big(\alpha \int_0^t \gamma(X_s)\,ds\Big)\,
    \mathbf{1}_{\{X_t=j\}}
    \,\Big|\,
    X_0=i
  \Big].
\end{equation}
We abbreviate $A(t) = \int_0^t \gamma(X_s)\,ds$ and write
$P(t) = P^{(0)}(t) = e^{Qt}$.

Differentiating with respect to $\alpha$ at $\alpha=0$ gives
\begin{align}
  U_{ij}(t)
  &:= \left.\frac{\partial}{\partial\alpha}
            P^{(\alpha)}_{ij}(t)\right|_{\alpha=0}
   = \mathbb{E}\big[A(t)\,\mathbf{1}_{\{X_t=j\}}\mid X_0=i\big],
  \label{eq:Udef}
  \\[4pt]
  W_{ij}(t)
  &:= \left.\frac{\partial^2}{\partial\alpha^2}
            P^{(\alpha)}_{ij}(t)\right|_{\alpha=0}
   = \mathbb{E}\big[A^2(t)\,\mathbf{1}_{\{X_t=j\}}\mid X_0=i\big].
  \label{eq:Wdef}
\end{align}

\section{Conditional mean and variance of integrated conductance}

By conditioning on the final state, we obtain
\begin{align}
  \mathbb{E}\big[A(t)\mid X_0=i, X_t=j\big]
  &= \frac{U_{ij}(t)}{P_{ij}(t)},
  \label{eq:meanAij}
  \\[4pt]
  \mathbb{E}\big[A^2(t)\mid X_0=i, X_t=j\big]
  &= \frac{W_{ij}(t)}{P_{ij}(t)}.
  \label{eq:secondmomentAij}
\end{align}
Therefore the conditional variance of the integrated conductance is
\begin{equation}
  \mathrm{Var}\big(A(t)\mid X_0=i, X_t=j\big)
  =
  \frac{W_{ij}(t)}{P_{ij}(t)}
  -
  \left(\frac{U_{ij}(t)}{P_{ij}(t)}\right)^2.
  \label{eq:varAij}
\end{equation}

In terms of the interval-averaged conductance
$\bar\gamma_{i\to j}(t) = A(t)/t$, we have
\begin{align}
  \mathbb{E}\big[\bar\gamma_{i\to j}(t)\big]
  &= \frac{1}{t}\,\frac{U_{ij}(t)}{P_{ij}(t)},
  \\[4pt]
  \mathrm{Var}\big(\bar\gamma_{i\to j}(t)\big)
  &= \frac{1}{t^2}
     \left[
       \frac{W_{ij}(t)}{P_{ij}(t)}
       -
       \left(\frac{U_{ij}(t)}{P_{ij}(t)}\right)^2
     \right].
\end{align}

\section{Block-matrix representation}

The derivatives in \eqref{eq:Udef}--\eqref{eq:Wdef} can be computed without
explicit diagonalization of $Q$ using matrix exponentials of block matrices.

\subsection{First derivative via a $2K\times 2K$ block matrix}

Consider the augmented generator
\begin{equation}
  \mathcal{Q}_2
  =
  \begin{pmatrix}
    Q & \Gamma\\
    0 & Q
  \end{pmatrix}
  \in \mathbb{R}^{2K\times 2K}.
\end{equation}
Its exponential has the block form
\begin{equation}
  e^{\mathcal{Q}_2 t}
  =
  \begin{pmatrix}
    e^{Qt} & F_1(t)\\
    0      & e^{Qt}
  \end{pmatrix},
\end{equation}
where
\begin{equation}
  F_1(t)
  =
  \int_0^t e^{Qu}\,\Gamma\,e^{Q(t-u)}\,du
  =
  \left.\frac{\partial}{\partial\alpha}
    e^{(Q+\alpha\Gamma)t}\right|_{\alpha=0}.
\end{equation}
Identifying $F_1(t)$ with the first derivative of the tilted semigroup, we
have
\begin{equation}
  U(t) = F_1(t),
\end{equation}
so the mean of $A(t)$ conditional on $(i,j)$ can be written as
\begin{equation}
  \mathbb{E}[A(t)\mid X_0=i,X_t=j]
  = \frac{F_{1,ij}(t)}{[e^{Qt}]_{ij}}.
\end{equation}

\subsection{Second derivative via a $3K\times 3K$ block matrix}

To access the second derivative simultaneously, define the $3K\times 3K$
block matrix
\begin{equation}
  \mathcal{Q}_3
  =
  \begin{pmatrix}
    Q & \Gamma & 0\\
    0 & Q      & \Gamma\\
    0 & 0      & Q
  \end{pmatrix}.
\end{equation}
Its exponential has the form
\begin{equation}
  e^{\mathcal{Q}_3 t}
  =
  \begin{pmatrix}
    e^{Qt} & F_1(t) & F_2(t)\\
    0      & e^{Qt} & F_1(t)\\
    0      & 0      & e^{Qt}
  \end{pmatrix},
\end{equation}
where $F_1(t)$ is as above, and
\begin{equation}
  F_2(t)
  =
  \int_0^t ds_1
  \int_0^{s_1} ds_2\;
  e^{Q s_2}\,\Gamma\,
  e^{Q(s_1-s_2)}\,\Gamma\,
  e^{Q(t-s_1)}.
\end{equation}

The second derivative of the tilted semigroup can be written as the symmetric
double integral
\begin{equation}
  W(t)
  =
  \left.\frac{\partial^2}{\partial\alpha^2}
    e^{(Q+\alpha\Gamma)t}\right|_{\alpha=0}
  =
  \int_0^t\!\!\int_0^t
   e^{Q s_<}\,\Gamma\,
   e^{Q(s_>-s_<)}\,\Gamma\,
   e^{Q(t-s_>)}\,ds_1\,ds_2,
\end{equation}
where $s_< = \min(s_1,s_2)$ and $s_> = \max(s_1,s_2)$. Since the square
$[0,t]\times[0,t]$ splits into two identical triangular regions
$s_2 < s_1$ and $s_1 < s_2$, we obtain
\begin{equation}
  W(t) = 2 F_2(t).
\end{equation}

Combining this with \eqref{eq:meanAij}--\eqref{eq:varAij}, we find
\begin{align}
  \mathbb{E}\big[A(t)\mid X_0=i,X_t=j\big]
  &= \frac{F_{1,ij}(t)}{[e^{Qt}]_{ij}},
  \\[4pt]
  \mathbb{E}\big[A^2(t)\mid X_0=i,X_t=j\big]
  &= \frac{2F_{2,ij}(t)}{[e^{Qt}]_{ij}},
  \\[4pt]
  \mathrm{Var}\big(A(t)\mid X_0=i,X_t=j\big)
  &=
  \frac{2F_{2,ij}(t)}{[e^{Qt}]_{ij}}
  -
  \left(\frac{F_{1,ij}(t)}{[e^{Qt}]_{ij}}\right)^2.
\end{align}

\end{document}
