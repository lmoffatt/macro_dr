\documentclass[11pt]{article}

\usepackage{amsmath,amssymb,amsfonts,bm}
\usepackage{geometry}
\geometry{margin=1in}

\title{MacroIR Interval Update\\[4pt]
Unified Boundary-State and Tilde Operator Specification}
\author{}
\date{}

\begin{document}

\maketitle

% Make the first section numbered as 0.
\setcounter{section}{-1}

\section{Scope and conventions}

We consider an ensemble of \(N_{\text{ch}}\) independent Markov channels with \(K\) microscopic states. Over an interval \([0,t]\) we observe a scalar interval-averaged macroscopic current \(\overline{y}_{0\to t}^{\text{obs}}\). We want the posterior mean and covariance of the macroscopic state at time \(t\), and the predictive mean and variance of the interval current.

\paragraph{Convention.} In what follows:
\begin{itemize}
  \item \((\boldsymbol{\mu}_0, \boldsymbol{\Sigma}_0)\) describe the per-channel occupancy statistics at the start of the interval:
    \begin{itemize}
      \item \(\mu_{0,i} = \mathbb{E}[\text{fraction of channels in state } i]\),
      \item \(\sum_i \mu_{0,i} = 1\).
    \end{itemize}
  \item Macroscopic means and variances are expressed per ion channel; total currents and variances scale with \(N_{\text{ch}}\).
\end{itemize}

You can convert to raw counts as \(\boldsymbol{n}_0 = N_{\text{ch}}\boldsymbol{\mu}_0\) if needed; all formulas simply scale accordingly.

\section{Indices, shapes, and operators}

\begin{itemize}
  \item \(K\): number of microscopic states.
  \item Indices:
    \begin{itemize}
      \item \(i_0, j_0\): start (time \(0\)) states.
      \item \(i_t, j_t\): end (time \(t\)) states.
      \item \(a,b\): generic state indices at time \(t\).
    \end{itemize}
  \item State vectors are columns:
    \[
      \boldsymbol{\mu}_0 \in \mathbb{R}^K,\quad
      \boldsymbol{\mu}^{\text{prior}}(t)\in\mathbb{R}^K,\ \text{etc.}
    \]
  \item State-property vectors are columns:
    \[
      \overline{\boldsymbol{\gamma}}_0 \in \mathbb{R}^K,\quad
      \boldsymbol{\sigma}^2_{\overline{\gamma}_0}\in\mathbb{R}^K,\ \text{etc.}
    \]
  \item State covariance matrices are indexed by two states at the same time:
    \[
      \boldsymbol{\Sigma}_0,\ \boldsymbol{\Sigma}^{\text{prop}}_{t}\in\mathbb{R}^{K\times K}
      \quad\text{(symmetric).}
    \]
  \item State transition and boundary-property matrices are indexed by the state at start time \(0\), \(i_0\), and at end time \(t\), \(i_t\):
    \begin{itemize}
      \item \(\mathbf{P}(t)\in\mathbb{R}^{K\times K}\) with entries \(P_{i_0\to i_t}(t)\) (asymmetric; \(\sum_{i_t} P_{i_0\to i_t}(t)=1\); row sums are \(1\)).
      \item Mean and variance-of-mean conductance matrices
      \(\overline{\mathbf{\Gamma}},\overline{\mathbf{V}}\in\mathbb{R}^{K\times K}\) indexed by \((i_0 \to i_t)\).
    \end{itemize}
  \item Elementwise (Hadamard) product: \(A\circ B\).
  \item \(\operatorname{diag}(x)\): diagonal matrix from vector \(x\).
  \item \(\operatorname{diag}(A)\): vector of diagonal entries of matrix \(A\).
  \item \(\mathbf{1}\): all-ones column vector in \(\mathbb{R}^K\).
\end{itemize}

\section{Inputs and precomputed microscopic objects}

For each interval \([0,t]\) we need:

\subsection{Macroscopic prior}

\begin{itemize}
  \item \(\boldsymbol{\mu}_0\),
  \item \(\boldsymbol{\Sigma}_0\) (covariance at time \(0\)).
\end{itemize}

\subsection{Markov dynamics}

\begin{itemize}
  \item Generator \(\mathbf{Q}\) (offline).
  \item Transition matrix
  \[
    \mathbf{P}(t) = e^{\mathbf{Q}t},\qquad
    P_{i_0\to i_t}(t) = [\mathbf{P}(t)]_{i_0 i_t}.
  \]
\end{itemize}

\subsection{Boundary-conditioned current statistics}

\begin{itemize}
  \item Mean interval current (per channel, conditioned on boundary states):
  \[
    \overline{\Gamma}_{i_0\to i_t}
    =
    \mathbb{E}\!\left[
      \frac{1}{t}\int_0 ^t y(\tau)\,\mathrm{d}\tau
      \,\middle|\,
      i_0,i_t
    \right].
  \]
  \item Variance contribution (intrinsic interval noise, per channel, conditioned on boundary states):
  \[
    \overline{V}_{i_0\to i_t}
    =
    \operatorname{Var}\!\left(
      \frac{1}{t}\int_0 ^t y(\tau)\,\mathrm{d}\tau
      \,\middle|\,
      i_0,i_t
    \right).
  \]
\end{itemize}

\subsection{Measurement setup}

\begin{itemize}
  \item Number of channels \(N_{\text{ch}}\).
  \item Instrument/binning noise:
  \[
    \epsilon^2_{0\to t} = \frac{\epsilon^2}{t} + \nu^2.
  \]
  \item Observation \(\overline{y}^{\text{obs}}_{0\to t}\).
\end{itemize}

\section{Boundary-state representation: the core insight}

\subsection{The boundary-state trick}

Introduce \emph{boundary states} \((i_0,i_t)\) representing start--end pairs. For each boundary pair, precompute:
\begin{itemize}
  \item Mean interval-averaged current: \(\overline{\Gamma}_{i_0 \to i_t}\).
  \item Intrinsic variance contribution: \(\overline{V}_{i_0 \to i_t}\).
\end{itemize}
These are arranged in \(K \times K\) matrices
\[
  \overline{\mathbf{\Gamma}},\quad
  \overline{\mathbf{V}},
\]
indexed by \((i_0,i_t)\).

\subsection{Boundary-state random variables}

For the ensemble, define the boundary count
\[
  N_{i_0\to i_t}
  =
  \text{number of channels that start in } i_0
  \text{ and end in } i_t.
\]

The \emph{boundary-state mean} is
\[
  \mu^{\text{prior}}_{0\to t,(i_0\to i_t)}
  =
  \mathbb{E}[N_{i_0\to i_t}]
  =
  (\boldsymbol{\mu}_0)_{i_0}\,P_{i_0\to i_t}(t).
\]

A convenient decomposition of the (per-channel-normalised) \emph{boundary-state covariance} is
\begin{align*}
  \Sigma^{\text{prior}}_{0\to t,(i_0\to i_t)(j_0\to j_t)}
  &=
  P_{i_0\to i_t}(t)
  \bigl[(\boldsymbol{\Sigma}_0)_{i_0 j_0}
        - \delta_{i_0 j_0}(\boldsymbol{\mu}_0)_{i_0}\bigr]
  P_{j_0\to j_t}(t) \\
  &\quad
  +\delta_{i_0 j_0}\delta_{i_t j_t}
  (\boldsymbol{\mu}_0)_{i_0}P_{i_0\to i_t}(t).
\end{align*}
This separates:
\begin{itemize}
  \item the propagated \emph{initial covariance} \(\boldsymbol{\Sigma}_0\), and
  \item the \emph{multinomial splitting noise} within each start state.
\end{itemize}

\subsection{Boundary-conditioned expectations}

Define the start-indexed conditional mean of the interval current:
\[
  (\overline{\gamma}_0)_{i_0}
  =
  \sum_{i_t} P_{i_0 \to i_t}(t)\,\overline{\Gamma}_{i_0 \to i_t}.
\]
Similarly, the intrinsic variance conditioned on \(i_0\):
\[
  (\sigma^2_{\overline{\gamma}_0})_{i_0}
  =
  \sum_{i_t} P_{i_0 \to i_t}(t)\,\overline{V}_{i_0 \to i_t}.
\]

\section{Boundary-lifted current statistics}

Define
\[
  \mathbf{G}
  =
  \overline{\mathbf{\Gamma}} \circ \mathbf{P}(t),\qquad
  G_{i_0 i_t}
  =
  \overline{\Gamma}_{i_0\to i_t} P_{i_0\to i_t}(t).
\]

\subsection{Start-conditioned mean interval current}

Component-wise:
\[
  (\overline{\gamma}_0)_{i_0}
  =
  \sum_{i_t} G_{i_0 i_t}.
\]
Matrix form:
\[
  \overline{\boldsymbol{\gamma}}_0 = \mathbf{G}\mathbf{1}.
\]

\subsection{Start-conditioned intrinsic variance}

Define
\[
  \mathbf{V}=\overline{\mathbf{V}}\circ\mathbf{P}(t),
\]
then
\[
  (\sigma^2_{\overline{\gamma}_0})_{i_0}
  =
  \sum_{i_t} V_{i_0 i_t}.
\]
Matrix form:
\[
  \boldsymbol{\sigma}^2_{\overline{\gamma}_0} = \mathbf{V}\mathbf{1}.
\]

\section{Predictive mean of the interval current}

The predictive mean interval current (per ensemble) is
\[
  \boxed{
    \overline{y}^{\text{pred}}_{0\to t}
    =
    N_{\text{ch}}\,
    \boldsymbol{\mu}_0^\top \overline{\boldsymbol{\gamma}}_0
  }.
\]

\section{The tilde operator}

We now define a unified family of operators that couple prior moments \((\boldsymbol{\mu}_0,\boldsymbol{\Sigma}_0)\), the transition matrix \(\mathbf{P}(t)\), and boundary matrices such as \(\overline{\mathbf{\Gamma}}\).

\subsection{Tilde over mean state: \texorpdfstring{\(\widetilde{\boldsymbol{\mu}_0}\)}{mu-tilde}}

The propagated mean at time \(t\) is
\[
  \boxed{
    \widetilde{\boldsymbol{\mu}_0}
    =
    \mathbf{P}(t)^\top \boldsymbol{\mu}_0
  }.
\]

\subsection{Tilde over state covariance: \texorpdfstring{\(\widetilde{\boldsymbol{\Sigma}_0}\)}{Sigma-tilde}}

The propagated covariance at time \(t\) is
\[
  \boxed{
    \widetilde{\boldsymbol{\Sigma}_0}
    =
    \mathbf{P}(t)^\top
    \bigl(\boldsymbol{\Sigma}_0-\operatorname{diag}(\boldsymbol{\mu}_0)\bigr)
    \mathbf{P}(t)
    +
    \operatorname{diag}\bigl(\mathbf{P}(t)^\top\boldsymbol{\mu}_0\bigr)
  }.
\]
This is the usual ``linear propagation plus process noise'' decomposition for multinomial splitting.

\subsection{Tilde over bilinear product:
\texorpdfstring{\(\widetilde{u^\top \Sigma w}\)}{uTSigmaw-tilde}}

Let \(\overline{\mathbf{U}}, \overline{\mathbf{W}}\in\mathbb{R}^{K\times K}\) be boundary-property matrices (analogous to \(\overline{\mathbf{\Gamma}}\)), and define the start-conditioned expectations
\[
  \overline{\boldsymbol{u}}_0
  =
  (\overline{\mathbf{U}}\circ\mathbf{P}(t))\mathbf{1},
  \qquad
  \overline{\boldsymbol{w}}_0
  =
  (\overline{\mathbf{W}}\circ\mathbf{P}(t))\mathbf{1}.
\]
Then the bilinear tilde is
\[
  \boxed{
    \widetilde{u^\top\Sigma w}
    =
    \overline{\boldsymbol{u}}_0^\top
    \bigl(\boldsymbol{\Sigma}_0-\operatorname{diag}(\boldsymbol{\mu}_0)\bigr)
    \overline{\boldsymbol{w}}_0
    +
    \boldsymbol{\mu}_0^\top
    \bigl[(\overline{\mathbf{U}}\circ\overline{\mathbf{W}}\circ\mathbf{P}(t))\mathbf{1}\bigr]
  }.
\]
Interpretation:
\begin{itemize}
  \item The first term is the action of the propagated initial covariance on the start-conditioned means.
  \item The second term is the additional variance from within-state multinomial splitting, with a \emph{single} factor of \(P_{i_0\to i_t}(t)\) per boundary pair.
\end{itemize}

\paragraph{Special case: interval current.}

For the interval current, set \(\overline{\mathbf{U}}=\overline{\mathbf{W}}=\overline{\mathbf{\Gamma}}\). Define
\[
  \mathbf{H}
  =
  (\overline{\mathbf{\Gamma}}\circ\overline{\mathbf{\Gamma}})\circ\mathbf{P}(t),
  \qquad
  H_{i_0 i_t}
  =
  \overline{\Gamma}_{i_0\to i_t}^2 P_{i_0\to i_t}(t).
\]
Then
\[
  \boxed{
    \widetilde{\gamma^\top\Sigma\gamma}
    =
    \overline{\boldsymbol{\gamma}}_0^\top
    \bigl(\boldsymbol{\Sigma}_0-\operatorname{diag}(\boldsymbol{\mu}_0)\bigr)
    \overline{\boldsymbol{\gamma}}_0
    +
    \boldsymbol{\mu}_0^\top
    \bigl[\mathbf{H}\mathbf{1}\bigr]
  }.
\]
Note carefully: the second term involves \(\overline{\Gamma}_{i_0\to i_t}^2 P_{i_0\to i_t}(t)\), not \(\overline{\Gamma}_{i_0\to i_t}^2 P_{i_0\to i_t}^2(t)\).

\subsection{Vector tilde: \texorpdfstring{\(\widetilde{u^\top\Sigma}\)}{uTSigma-tilde}}

The vector tilde gives the cross-covariance between the state at time \(t\) and a boundary-weighted scalar.

For \(\overline{\mathbf{U}}\) as above and \(\overline{\boldsymbol{u}}_0\) defined as in the previous subsection:
\[
  \boxed{
    \widetilde{u^\top\Sigma}
    =
    \mathbf{P}(t)^\top
    \bigl(\boldsymbol{\Sigma}_0-\operatorname{diag}(\boldsymbol{\mu}_0)\bigr)
    \overline{\boldsymbol{u}}_0
    +
    (\overline{\mathbf{U}}\circ\mathbf{P}(t))^\top \boldsymbol{\mu}_0
  }.
\]

For the interval current we set \(\overline{\mathbf{U}}=\overline{\mathbf{\Gamma}}\), and define
\[
  \boxed{
    \mathbf{g}
    =
    \widetilde{\gamma^\top\Sigma}
    =
    \mathbf{P}(t)^\top
    \bigl(\boldsymbol{\Sigma}_0-\operatorname{diag}(\boldsymbol{\mu}_0)\bigr)
    \overline{\boldsymbol{\gamma}}_0
    +
    \mathbf{G}^\top\boldsymbol{\mu}_0
  }.
\]

\section{Predictive variance of the interval current}

The predictive variance (per ensemble) decomposes as measurement noise plus contributions from state uncertainty and intrinsic interval noise:
\[
  \boxed{
    \sigma^2_{\overline{y}^{\text{pred}}}
    =
    \epsilon^2_{0\to t}
    +
    N_{\text{ch}}\,
    \widetilde{\gamma^\top\Sigma\gamma}
    +
    N_{\text{ch}}\,
    \boldsymbol{\mu}_0^\top\boldsymbol{\sigma}^2_{\overline{\gamma}_0}
  }.
\]

\section{Propagation to time \texorpdfstring{\(t\)}{t}}

\subsection{Mean}

Using the tilde from Section~5.1:
\[
  \boxed{
    \boldsymbol{\mu}^{\text{prop}}(t)
    =
    \widetilde{\boldsymbol{\mu}_0}
    =
    \mathbf{P}(t)^\top\boldsymbol{\mu}_0
  }.
\]

\subsection{Covariance}

Using the tilde from Section~5.2:
\[
  \boxed{
    \boldsymbol{\Sigma}^{\text{prop}}(t)
    =
    \widetilde{\boldsymbol{\Sigma}_0}
    =
    \mathbf{P}(t)^\top
    \bigl(\boldsymbol{\Sigma}_0-\operatorname{diag}(\boldsymbol{\mu}_0)\bigr)
    \mathbf{P}(t)
    +
    \operatorname{diag}\bigl(\boldsymbol{\mu}^{\text{prop}}(t)\bigr)
  }.
\]

\section{Measurement update}

Let
\[
  \delta =
  \overline{y}^{\text{obs}}_{0\to t}
  -
  \overline{y}^{\text{pred}}_{0\to t}.
\]

\subsection{Mean update}

Gaussian conditioning (scalar observation):
\[
  \boxed{
    \boldsymbol{\mu}^{\text{post}}(t)
    =
    \boldsymbol{\mu}^{\text{prop}}(t)
    +
    \frac{\mathbf{g}\,\delta}{
      \sigma^2_{\overline{y}^{\text{pred}}}
    }
  }.
\]

\subsection{Covariance update}

Rank-1 downdate:
\[
  \boxed{
    \boldsymbol{\Sigma}^{\text{post}}(t)
    =
    \boldsymbol{\Sigma}^{\text{prop}}(t)
    -
    \frac{\mathbf{g}\mathbf{g}^\top}{
      \sigma^2_{\overline{y}^{\text{pred}}}
    }
  }.
\]

\section{Summary of workflow}

\begin{enumerate}
  \item Inputs:
  \(\boldsymbol{\mu}_0,\boldsymbol{\Sigma}_0, \mathbf{P}(t),
   \overline{\mathbf{\Gamma}}, \overline{\mathbf{V}},
   N_{\text{ch}}, \epsilon^2, \nu^2,
   \overline{y}^{\text{obs}}_{0\to t}\).

  \item Compute:
  \begin{itemize}
    \item \(\mathbf{G}=\overline{\mathbf{\Gamma}}\circ\mathbf{P}(t)\).
    \item \(\overline{\boldsymbol{\gamma}}_0=\mathbf{G}\mathbf{1}\).
    \item \(\mathbf{V}=\overline{\mathbf{V}}\circ\mathbf{P}(t)\).
    \item \(\boldsymbol{\sigma}^2_{\overline{\gamma}_0}=\mathbf{V}\mathbf{1}\).
  \end{itemize}

  \item Predictive current:
  \[
    \overline{y}^{\text{pred}}_{0\to t}
    =
    N_{\text{ch}}\,
    \boldsymbol{\mu}_0^\top\overline{\boldsymbol{\gamma}}_0,
  \]
  \[
    \sigma^2_{\overline{y}^{\text{pred}}}
    =
    \epsilon^2_{0\to t}
    +
    N_{\text{ch}}\widetilde{\gamma^\top\Sigma\gamma}
    +
    N_{\text{ch}}\,
    \boldsymbol{\mu}_0^\top\boldsymbol{\sigma}^2_{\overline{\gamma}_0}.
  \]

  \item Cross-covariance:
  \(\mathbf{g}=\widetilde{\gamma^\top\Sigma}\) from Section~5.4.

  \item Propagate:
  \begin{itemize}
    \item \(\boldsymbol{\mu}^{\text{prop}}(t)=\mathbf{P}(t)^\top\boldsymbol{\mu}_0\).
    \item \(\boldsymbol{\Sigma}^{\text{prop}}(t)=\mathbf{P}(t)^\top(\boldsymbol{\Sigma}_0-\operatorname{diag}(\boldsymbol{\mu}_0))\mathbf{P}(t)+\operatorname{diag}(\boldsymbol{\mu}^{\text{prop}}(t))\).
  \end{itemize}

  \item Update:
  \begin{itemize}
    \item \(\boldsymbol{\mu}^{\text{post}}(t)=\boldsymbol{\mu}^{\text{prop}}(t)+\mathbf{g}\,\delta/\sigma^2_{\overline{y}^{\text{pred}}}\).
    \item \(\boldsymbol{\Sigma}^{\text{post}}(t)=\boldsymbol{\Sigma}^{\text{prop}}(t)-\mathbf{g}\mathbf{g}^\top/\sigma^2_{\overline{y}^{\text{pred}}}\).
  \end{itemize}
\end{enumerate}

All steps use only \(K\)-vector and \(K\times K\)-matrix objects; no \(K^2\times K^2\) boundary covariance is ever instantiated.

\section{Numerical notes and invariants}

\begin{itemize}
  \item \(\boldsymbol{\Sigma}_0\) and \(\boldsymbol{\Sigma}^{\text{prop}}(t)\) should be symmetric; small asymmetries from floating point should be symmetrized explicitly.
  \item The predictive variance \(\sigma^2_{\overline{y}^{\text{pred}}}\) must be strictly positive; in practice, impose a lower bound to avoid division by zero.
  \item The posterior covariance \(\boldsymbol{\Sigma}^{\text{post}}(t)\) is obtained by a rank-1 subtraction; numerical positive-semidefiniteness violations can be mitigated by enforcing symmetry and clipping tiny negative eigenvalues in post-processing.
  \item Computational complexity: matrix--matrix multiplications are \(O(K^3)\) (dominated by covariance propagation); Hadamard products and tilde expressions are \(O(K^2)\).
\end{itemize}

\appendix

\section{Pedagogical narrative (optional)}

This appendix compresses the conceptual story of MacroIR; it is not needed for implementation, but can help with intuition.

\subsection{Why boundary states?}

The interval-averaged current of a channel depends on its entire trajectory over \([0,t]\), not just its start or end state. Directly integrating over all trajectories is intractable.

MacroIR uses the following trick:
\begin{enumerate}
  \item Introduce boundary states \((i_0,i_t)\).
  \item For each boundary pair, precompute:
    \begin{itemize}
      \item mean interval current \(\overline{\Gamma}_{i_0\to i_t}\),
      \item variance \(\overline{V}_{i_0\to i_t}\).
    \end{itemize}
  \item Note that, given initial counts, channels from each \(i_0\) split multinomially into end states with probabilities \(P_{i_0\to i_t}(t)\). The random boundary counts \(N_{i_0\to i_t}\) therefore contain all the trajectory information that matters for the interval current.
\end{enumerate}

The macroscopic interval current can then be written as a linear functional of the boundary counts plus additive noise:
\[
  \overline{y}_{0\to t}
  \approx
  \sum_{i_0,i_t}
  \overline{\Gamma}_{i_0\to i_t}N_{i_0\to i_t}
  +
  \text{noise}.
\]

\subsection{Why the tilde operator?}

Naively, the boundary count covariance is a \(K^2\times K^2\) object, which we never want to build explicitly. Instead, observe:
\begin{itemize}
  \item The prior information lives in the state space at \(t=0\): \((\boldsymbol{\mu}_0,\boldsymbol{\Sigma}_0)\).
  \item The microscopic physics over \([0,t]\) is fully captured by:
    \begin{itemize}
      \item the transition matrix \(\mathbf{P}(t)\),
      \item the boundary tables \(\overline{\mathbf{\Gamma}},\overline{\mathbf{V}}\).
    \end{itemize}
\end{itemize}

The tilde operator is the algebraic mechanism that:
\begin{enumerate}
  \item \emph{Lifts} a state-space vector (such as \(\overline{\boldsymbol{\gamma}}_0\)) into the boundary space via the boundary matrices.
  \item \emph{Modulates} it with the transition probabilities \(\mathbf{P}(t)\) and the initial covariance \(\boldsymbol{\Sigma}_0\).
  \item \emph{Collapses} back to either:
    \begin{itemize}
      \item a scalar \(\widetilde{u^{\top}\Sigma w}\), or
      \item a vector \(\widetilde{u^{\top}\Sigma}\),
    \end{itemize}
    without ever explicitly constructing the full boundary covariance.
\end{enumerate}

Conceptually:
\begin{itemize}
  \item \(\widetilde{\gamma^{\top}\Sigma\gamma}\) is the scalar one would obtain by forming the boundary covariance and applying the quadratic form defined by the boundary current weights.
  \item \(\widetilde{\gamma^{\top}\Sigma}\) is the cross-covariance between the macroscopic state at time \(t\) and the interval current.
\end{itemize}
The lift--modulate--collapse pattern is the same in both cases; only the rank of the underlying algebraic expression differs, just like comparing \(v^{\top} A\) (vector) with \(v^{\top}Aw\) (scalar) in ordinary matrix algebra.

\subsection{Why the update looks like a scalar Kalman filter}

Once you have:
\begin{itemize}
  \item a prior Gaussian state at time \(t\),
  \((\boldsymbol{\mu}^{\text{prop}}(t),\boldsymbol{\Sigma}^{\text{prop}}(t))\);
  \item a scalar observation \(\overline{y}^{\text{obs}}_{0\to t}\);
  \item its predictive mean and variance
  \(\overline{y}^{\text{pred}}_{0\to t},\sigma^2_{\overline{y}^{\text{pred}}}\);
  \item a cross-covariance vector \(\mathbf{g}\);
\end{itemize}
you are exactly in the setting of a one-dimensional Kalman update in a \(K\)-dimensional state space:
\begin{itemize}
  \item \(\mathbf{g}\) plays the role of ``measurement vector times prior covariance'',
  \item \(\sigma^2_{\overline{y}^{\text{pred}}}\) is the total effective measurement variance.
\end{itemize}

The update
\[
  \boldsymbol{\mu}^{\text{post}}
  =
  \boldsymbol{\mu}^{\text{prop}}
  +
  \frac{\mathbf{g}}{\sigma^2_{\overline{y}^{\text{pred}}}}\delta,
  \qquad
  \boldsymbol{\Sigma}^{\text{post}}
  =
  \boldsymbol{\Sigma}^{\text{prop}}
  -
  \frac{\mathbf{g}\mathbf{g}^{\top}}{\sigma^2_{\overline{y}^{\text{pred}}}},
\]
is precisely the Gaussian conditioning formula for a joint Gaussian variable
\((\boldsymbol{N}_{\text{ch}}(t),\overline{y}_{0\to t})\).

MacroIR's distinctive feature is not the Kalman structure itself, but that \(\mathbf{g}\) and \(\sigma^2_{\overline{y}^{\text{pred}}}\) are computed from \emph{interval physics} via boundary matrices and the unified tilde operator, rather than from a naive instant-state observation model.

\end{document}
